\documentclass[../thesis.tex]{subfiles}

\begin{document}

Chương này sẽ mô tả cách sử dụng Cloud Translation API để dịch văn bản thông qua dịch vụ Translate.

% \section{Phương thức GET}
% Sau đây là cách sử dụng phương thức GET của HTTP, ta copy URL kèm theo các tham số và dán vào trình duyệt. URL có dạng:
% 
% \begin{lstlisting}[numbers=none, frame=single,xleftmargin=0.15cm,xrightmargin=0.15cm]
% https://translation.googleapis.com/language/translate/v2
% \end{lstlisting}

% Khi sử dụng phương thức này, URL kèm theo các tham số phải ít hơn 2000 ký tự. 
% 
% Ví dụ: dịch một chuỗi ``He is a student'' sang tiếng Séc thì URL sẽ có dạng
% 
% \begin{lstlisting}[numbers=none, frame=single,xleftmargin=0.15cm,xrightmargin=0.15cm]
% https://translation.googleapis.com/language/translate/v2?q=He is a student&target=cs&key=MY_API_KEY
% \end{lstlisting}
% với \lstinline{MY_API_KEY} là API key mà ta đã tạo.
% 
% Kết quả nhận được ở dạng JSON như sau:
% 
% \begin{lstlisting}[numbers=none, frame=single,xleftmargin=0.15cm,xrightmargin=0.15cm]
% {
%   "data": {
%     "translations": [
%       {
%         "translatedText": "On je student",
%         "detectedSourceLanguage": "en"
%       }
%     ]
%   }
% }
% \end{lstlisting}

\section{HTTP request}

Lệnh POST được định nghĩa như sau:

\begin{lstlisting}[numbers=none, frame=single,xleftmargin=0.15cm,xrightmargin=0.15cm]
POST https://translation.googleapis.com/language/translate/v2?key=YOUR_API_KEY
\end{lstlisting}

Các tham số:

\begin{center}
\begin{tabularx}{\textwidth}{|p{0.15\textwidth}|X|}
\hline
\lstinline{q} & \textbf{string}

\textit{Bắt buộc}. Đoạn văn bản cần dịch. Lặp lại tham số này nếu có nhiều đoạn.\\
\hline
\lstinline{target} & \textbf{string}

\textit{Bắt buộc}. Mã ISO-639-1 của ngôn ngữ mà ta muốn dịch sang.\\
\hline
\lstinline{format} & \textbf{string}

Định dạng của đoạn văn bản cần dịch. Có giá trị là \lstinline{html} nếu là HTML và \lstinline{text} nếu là plain-text.\\
\hline
\lstinline{source} & \textbf{string}

Mã ISO-639-1 của \lstinline{q}. Nếu không cung cấp tham số này, API sẽ tự động phát hiện ngôn ngữ.\\
\hline
\lstinline{model} & \textbf{string}

Mô hình dịch. Nếu có giá trị là \lstinline{base} thì API sẽ sử dụng mô hình Phrase-Based Machine Translation (PBMT), hoặc \lstinline{nmt} cho mô hình Neural Machine Translation (NMT). Nếu bỏ qua tham số này, mô hình nmt sẽ được sử dụng.

Mô hình NMT chỉ hỗ trợ khi ngôn ngữ nguồn hoặc đích là tiếng Anh (en).\\
\hline
\lstinline{key} & \textbf{string}

API key. Nếu sử dụng OAuth 2.0 thì không cần tham số này.\\
\hline
\end{tabularx}
\end{center}

Ta sử dụng nhiều tham số \lstinline{q} để định nghĩa các chuỗi cần dịch và tham số \lstinline{target} để định nghĩa ngôn ngữ đích. Các tham số này đặt trong chuỗi JSON:

\begin{lstlisting}[numbers=none, frame=single,xleftmargin=0.15cm,xrightmargin=0.15cm]
{
  'q': 'Hello world',
  'q': 'My name is Jeff',
  'target': 'de'
}
\end{lstlisting}

Nếu thành công, HTTP trả về mã trạng thái là \lstinline{200 OK} thì kết quả nhận được như sau:

\begin{lstlisting}[numbers=none, frame=single,xleftmargin=0.15cm,xrightmargin=0.15cm]
{
  "data": {
    "translations": [
      {
        "translatedText": "Hallo Welt",
        "detectedSourceLanguage": "en"
      },
      {
        "translatedText": "Mein Name ist Jeff",
        "detectedSourceLanguage": "en"
      }
    ]
  }
} 
\end{lstlisting}

\section{Dịch vụ Translate trong Java}
Để sử dụng dịch vụ Translate trong Java, ta thực hiện theo quy trình sau:
\begin{enumerate}
  \item Tạo một đối tượng thuộc lớp \lstinline{Translate.Builder}.
  \item Sử dụng phương thức \lstinline{build()} của đối tượng \lstinline{Builder} vừa tạo để khởi tạo một dịch vụ \lstinline{Translate}.
  \item Tạo một yêu cầu (đối tượng thuộc lớp \lstinline{Translations.List}) và sử dụng các setter để định các tham số cho lệnh POST.
  \item Thực thi yêu cầu bằng phương thức \lstinline{execute()}. Kết quả trả về ở dạng JSON, thuộc lớp \lstinline{TranslationsListResponse}.
  \item Dùng phương thức \lstinline{getTranslations()} nhận về một \lstinline{List} các đối tượng thuộc lớp \lstinline{TranslationResource}.
  \item Dùng các phương thức \lstinline{getDetectedSourceLanguage()} và \lstinline{getTranslatedText()} để nhận về các chuỗi đã dịch và ngôn ngữ được phát hiện.
\end{enumerate}

\section*{Tham khảo}

\begin{enumerate}
  \item Documentation - Translating Text: https://goo.gl/tQgYgy
  \item Google Cloud Translation API Java Docs: https://goo.gl/bnU3ES
\end{enumerate}

\end{document}