\documentclass[../thesis.tex]{subfiles}

\begin{document}

Chương này sẽ mô tả cách sử dụng Cloud Translation API để dịch văn bản thông qua dịch vụ Translate.

\section{HTTP request}

Lệnh \lstinline{POST} được định nghĩa như sau:

\begin{lstlisting}[style=link]
POST https://translation.googleapis.com/language/translate/v2?key=YOUR_API_KEY
\end{lstlisting}

Các tham số:

\begin{center}
\begin{tabularx}{\textwidth}{|p{0.15\textwidth}|X|}
\hline
\lstinline{q} & \textbf{string}

\textit{Bắt buộc}. Đoạn văn bản cần dịch. Lặp lại tham số này nếu có nhiều đoạn.\\
\hline
\lstinline{target} & \textbf{string}

\textit{Bắt buộc}. Mã ISO-639-1 của ngôn ngữ mà ta muốn dịch sang.\\
\hline
\lstinline{format} & \textbf{string}

Định dạng của đoạn văn bản cần dịch. Có giá trị là \lstinline{html} nếu là HTML và \lstinline{text} nếu là plain-text.\\
\hline
\lstinline{source} & \textbf{string}

Mã ISO-639-1 của \lstinline{q}. Nếu không cung cấp tham số này, API sẽ tự động phát hiện ngôn ngữ.\\
\hline
\lstinline{model} & \textbf{string}

Mô hình dịch. Nếu có giá trị là \lstinline{base} thì API sẽ sử dụng mô hình Phrase-Based Machine Translation (PBMT), hoặc \lstinline{nmt} cho mô hình Neural Machine Translation (NMT). Nếu bỏ qua tham số này, mô hình nmt sẽ được ưu tiên sử dụng.

Mô hình NMT chỉ hỗ trợ khi ngôn ngữ nguồn hoặc đích là tiếng Anh (en).\\
\hline
\lstinline{key} & \textbf{string}

API key. Nếu sử dụng OAuth 2.0 thì không cần tham số này.\\
\hline
\end{tabularx}
\end{center}

\section{Kết quả trả về}
Nếu thành công, kết quả trả về ở dạng JSON như sau:
\begin{lstlisting}[style=link]
{
	"data": {
		object(TranslateTextResponseList)
	},
}
\end{lstlisting}

Các trường dữ liệu:
\begin{center}
\begin{tabularx}{\textwidth}{|p{0.15\textwidth}|X|}
\hline
\lstinline{data} & \lstinline{object(TranslateTextResponseList)}

Danh sách các kết quả dịch cho mỗi giá trị của \lstinline{q}.\\
\hline
\end{tabularx}
\end{center}

\section{TranslateTextResponseList}
Mỗi phần tử trong danh sách kết quả là một mảng các thông tin liên quan đến giá trị \lstinline{q} đã được dịch. Nó có dạng JSON như sau:

\begin{lstlisting}[style=link]
{
	"translations": [
		array
	],
}
\end{lstlisting}

Các trường dữ liệu:
\begin{center}
\begin{tabularx}{\textwidth}{|p{0.2\textwidth}|X|}
\hline
\lstinline{translations[]} & \lstinline{array (TranslateTextResponseTranslation)}

Kết quả dịch.\\
\hline
\end{tabularx}
\end{center}

\section{Dịch văn bản}
\subsection{Dịch chuỗi văn bản}
Sử dụng tham số \lstinline{q} để định các chuỗi văn bản cần dịch. HTTP request sau dịch 2 chuỗi sang tiếng Phần Lan:
\begin{lstlisting}[style=link]
 POST https://translation.googleapis.com/language/translate/v2?key=YOUR_API_KEY
\end{lstlisting}
\begin{lstlisting}[style=link]
{
	'q': 'Hello world',
	'q': 'My name is Jeff',
	'target': 'fi'
}
\end{lstlisting}

Nếu thành công, HTTP trả về mã trạng thái là \lstinline{200 OK} thì kết quả nhận được là một chuỗi JSON có dạng như sau:

\begin{lstlisting}[style=link]
{
	"data": {
		"translations": [
			{
				"translatedText": "Hei maailma",
				"detectedSourceLanguage": "en"
			},
			{
				"translatedText": "Nimeni on Jeff",
				"detectedSourceLanguage": "en"
			}
		]
	}
}
\end{lstlisting}

\subsection{Sử dụng một mô hình cụ thể}
Mặc định, mô hình NMT được ưu tiên sử dụng. Trong trường hợp không sử dụng được mô hình NMT, mô hình PBMT được sử dụng. Ta sẽ thử dịch cùng một đoạn văn bản sử dụng 2 mô hình này và so sánh kết quả.

Sử dụng mô hình NMT, HTTP request có các tham số sau:
\begin{lstlisting}[style=link]
{
	'q': 'Education is the process of facilitating learning, or the acquisition of knowledge, skills, values, beliefs, and habits. Educational methods include storytelling, discussion, teaching, training, and directed research',
	'target': 'vi',
	'model': 'nmt'
}
\end{lstlisting}

Bản dịch nhận được là: ``Giáo dục là quá trình tạo điều kiện thuận lợi cho học tập, hoặc thu thập kiến thức, kỹ năng, giá trị, niềm tin và thói quen. Các phương pháp giáo dục bao gồm kể chuyện, thảo luận, giảng dạy, đào tạo và nghiên cứu trực tiếp''.

Sử dụng mô hình PBMT, HTTP request có các tham số sau:
\begin{lstlisting}[style=link]
{
	'q': 'Education is the process of facilitating learning, or the acquisition of knowledge, skills, values, beliefs, and habits. Educational methods include storytelling, discussion, teaching, training, and directed research',
	'target': 'vi',
	'model': 'base'
}
\end{lstlisting}

Bản dịch nhận được là: "Giáo dục là quá trình tạo điều kiện học tập, hoặc tiếp thu kiến thức, kỹ năng, giá trị, niềm tin, và thói quen. phương pháp giáo dục bao gồm kể chuyện, thảo luận, giảng dạy, đào tạo và nghiên cứu đạo".

Có thể thấy rằng mô hình NMT cho kết quả dịch thuật chính xác hơn và mang ý nghĩa dịch thuật cao hơn.

\section{Dịch vụ Translate trong Java}
Để sử dụng dịch vụ Translate trong Java, ta thực hiện theo quy trình sau:
\begin{enumerate}
  \item Tạo một đối tượng thuộc lớp \lstinline{Translate.Builder}.
  \item Sử dụng phương thức \lstinline{build()} của đối tượng \lstinline{Builder} vừa tạo để khởi tạo một dịch vụ \lstinline{Translate}.
  \item Tạo một yêu cầu (đối tượng thuộc lớp \lstinline{Translations.List}) và sử dụng các setter để định các tham số cho lệnh \lstinline{POST}.
  \item Thực thi yêu cầu bằng phương thức \lstinline{execute()}. Kết quả trả về ở dạng JSON, thuộc lớp \lstinline{TranslationsListResponse}.
  \item Dùng phương thức \lstinline{getTranslations()} nhận về một \lstinline{List} các đối tượng thuộc lớp \lstinline{TranslationResource}.
  \item Dùng các phương thức \lstinline{getDetectedSourceLanguage()} và \lstinline{getTranslatedText()} để nhận về các chuỗi đã dịch và ngôn ngữ được phát hiện (nếu tham số \lstinline{source} chưa được xác định).
\end{enumerate}

\section*{Tham khảo}

\begin{enumerate}
  \item Documentation - Translating Text: https://goo.gl/Qbziwx
  \item REST Reference - translate: https://goo.gl/uRtqmi
  \item Google Cloud Translation API v2 Java Docs: https://goo.gl/bnU3ES
\end{enumerate}

\end{document}